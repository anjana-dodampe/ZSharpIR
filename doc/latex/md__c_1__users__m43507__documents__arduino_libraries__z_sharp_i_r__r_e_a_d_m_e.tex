Arduino Infra Red Sharp Lib

Based on an original work of Dr. Marcal Casas-\/\+Cartagena .


\begin{DoxyEnumerate}
\item Perform 25 reading of analog pin (Nb samples can be changed in .h)
\item Sort values
\item Convert median value to cm
\end{DoxyEnumerate}

\subsection*{Usage}


\begin{DoxyItemize}
\item \#include $<$Sharp\+I\+R.\+h$>$
\item Sharp\+IR sharp(ir\+\_\+analog\+\_\+pin, model);
\item int dist = sharp.\+distance();
\end{DoxyItemize}

Model \+:
\begin{DoxyItemize}
\item G\+P2\+Y0\+A02\+Y\+K0F --$>$ \char`\"{}20150\char`\"{}
\item G\+P2\+Y0\+A21\+YK --$>$ \char`\"{}1080\char`\"{}
\item G\+P2\+Y0\+A710\+K0F --$>$ \char`\"{}100500\char`\"{}
\item G\+P2\+Y\+A41\+S\+K0F --$>$ \char`\"{}430\char`\"{}
\end{DoxyItemize}

\subsection*{Sharp IR Volt Centimeter conversion}

\subsubsection*{G\+P2\+Y0\+A02\+Y\+K0F}

\paragraph*{Model\+: \char`\"{}20150\char`\"{} \mbox{[}20cm to 150cm\mbox{]}}

\tabulinesep=1mm
\begin{longtabu} spread 0pt [c]{*{2}{|X[-1]}|}
\hline
\rowcolor{\tableheadbgcolor}\textbf{ Volt  }&\textbf{ Distance   }\\\cline{1-2}
\endfirsthead
\hline
\endfoot
\hline
\rowcolor{\tableheadbgcolor}\textbf{ Volt  }&\textbf{ Distance   }\\\cline{1-2}
\endhead
2,8  &15   \\\cline{1-2}
2,5  &20   \\\cline{1-2}
2  &30   \\\cline{1-2}
1,55  &40   \\\cline{1-2}
1,24  &50   \\\cline{1-2}
1,05  &60   \\\cline{1-2}
0,905  &70   \\\cline{1-2}
0,82  &80   \\\cline{1-2}
0,7  &90   \\\cline{1-2}
0,66  &100   \\\cline{1-2}
0,6  &110   \\\cline{1-2}
0,55  &120   \\\cline{1-2}
0,5  &130   \\\cline{1-2}
0,455  &140   \\\cline{1-2}
0,435  &150   \\\cline{1-2}
\end{longtabu}


Using MS Excel, we can calculate function (For distance $>$ 15cm) \+:

Distance = 60.\+374 X P\+OW(Volt , -\/1.\+16)

\subsubsection*{G\+P2\+Y0\+A21\+YK}

\paragraph*{Model\+: \char`\"{}1080\char`\"{} \mbox{[}10cm to 80cm\mbox{]}}

\tabulinesep=1mm
\begin{longtabu} spread 0pt [c]{*{2}{|X[-1]}|}
\hline
\rowcolor{\tableheadbgcolor}\textbf{ Volt  }&\textbf{ Distance   }\\\cline{1-2}
\endfirsthead
\hline
\endfoot
\hline
\rowcolor{\tableheadbgcolor}\textbf{ Volt  }&\textbf{ Distance   }\\\cline{1-2}
\endhead
2,6  &10   \\\cline{1-2}
2,1  &12   \\\cline{1-2}
1,85  &14   \\\cline{1-2}
1,65  &15   \\\cline{1-2}
1,5  &18   \\\cline{1-2}
1,39  &20   \\\cline{1-2}
1,15  &25   \\\cline{1-2}
0,98  &30   \\\cline{1-2}
0,85  &35   \\\cline{1-2}
0,75  &40   \\\cline{1-2}
0,67  &45   \\\cline{1-2}
0,61  &50   \\\cline{1-2}
0,59  &55   \\\cline{1-2}
0,55  &60   \\\cline{1-2}
0,5  &65   \\\cline{1-2}
0,48  &70   \\\cline{1-2}
0,45  &75   \\\cline{1-2}
0,42  &80   \\\cline{1-2}
\end{longtabu}


Using MS Excel, we can calculate function (For distance $>$ 10cm) \+:

Distance = 29.\+988 X P\+OW(Volt , -\/1.\+173)

\subsubsection*{G\+P2\+Y0\+A710\+K0F}

\paragraph*{Model\+: \char`\"{}100500\char`\"{} \mbox{[}100cm to 500cm\mbox{]}}

Based on the S\+H\+A\+RP datasheet we can calculate the linear function\+: {\ttfamily y = 137500x + 1125} which gives us\+: {\ttfamily 1 / ((Volt -\/ 1125) / 137500) = distance\+\_\+in\+\_\+cm} (For distance $>$ 100cm)

\subsubsection*{G\+P2\+Y\+A41\+S\+K0F ( $<$=$>$ G\+P2\+D120 )}

\paragraph*{Model\+: \char`\"{}430\char`\"{} \mbox{[}4cm to 30cm\mbox{]}}

Based on the S\+H\+A\+RP datasheet we can calculate the function (For distance $>$ 3cm) \+:

Distance = 12.\+08 X P\+OW(Volt , -\/1.\+058) 